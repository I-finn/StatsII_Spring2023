\documentclass[12pt,letterpaper]{article}
\usepackage{graphicx,textcomp}
\usepackage{natbib}
\usepackage{setspace}
\usepackage{fullpage}
\usepackage{color}
\usepackage[reqno]{amsmath}
\usepackage{amsthm}
\usepackage{fancyvrb}
\usepackage{amssymb,enumerate}
\usepackage[all]{xy}
\usepackage{endnotes}
\usepackage{lscape}
\newtheorem{com}{Comment}
\usepackage{float}
\usepackage{hyperref}
\newtheorem{lem} {Lemma}
\newtheorem{prop}{Proposition}
\newtheorem{thm}{Theorem}
\newtheorem{defn}{Definition}
\newtheorem{cor}{Corollary}
\newtheorem{obs}{Observation}
\usepackage[compact]{titlesec}
\usepackage{dcolumn}
\usepackage{tikz}
\usetikzlibrary{arrows}
\usepackage{multirow}
\usepackage{xcolor}
\newcolumntype{.}{D{.}{.}{-1}}
\newcolumntype{d}[1]{D{.}{.}{#1}}
\definecolor{light-gray}{gray}{0.65}
\usepackage{url}
\usepackage{listings}
\usepackage{color}

\usepackage{verbatim}
%\usepackage[utf8]{inputenc}
\usepackage[T1]{fontenc}

\definecolor{codegreen}{rgb}{0,0.6,0}
\definecolor{codegray}{rgb}{0.5,0.5,0.5}
\definecolor{codepurple}{rgb}{0.58,0,0.82}
\definecolor{backcolour}{rgb}{0.95,0.95,0.92}

\lstdefinestyle{mystyle}{
	backgroundcolor=\color{backcolour},   
	commentstyle=\color{codegreen},
	keywordstyle=\color{magenta},
	numberstyle=\tiny\color{codegray},
	stringstyle=\color{codepurple},
	basicstyle=\footnotesize,
	breakatwhitespace=false,         
	breaklines=true,                 
	captionpos=b,                    
	keepspaces=true,                 
	numbers=left,                    
	numbersep=5pt,                  
	showspaces=false,                
	showstringspaces=false,
	showtabs=false,                  
	tabsize=2
}
\lstset{style=mystyle}
\newcommand{\Sref}[1]{Section~\ref{#1}}
\newtheorem{hyp}{Hypothesis}

\title{Problem Set 2}
\date{Due: February 19, 2023}
\author{Applied Stats II}


\begin{document}
	\maketitle
	\section*{Instructions}
	\begin{itemize}
		\item Please show your work! You may lose points by simply writing in the answer. If the problem requires you to execute commands in \texttt{R}, please include the code you used to get your answers. Please also include the \texttt{.R} file that contains your code. If you are not sure if work needs to be shown for a particular problem, please ask.
		\item Your homework should be submitted electronically on GitHub in \texttt{.pdf} form.
		\item This problem set is due before 23:59 on Sunday February 19, 2023. No late assignments will be accepted.
	%	\item Total available points for this homework is 80.
	\end{itemize}

	
	%	\vspace{.25cm}
	
	Code in \texttt{PS2\_ImeldaFinn.R}

%\section*{Question 1} %(20 points)}
%\vspace{.25cm}
\noindent We're interested in what types of international environmental agreements or policies people support (\href{https://www.pnas.org/content/110/34/13763}{Bechtel and Scheve 2013)}. So, we asked 8,500 individuals whether they support a given policy, and for each participant, we vary the (1) number of countries that participate in the international agreement and (2) sanctions for not following the agreement. \\

\noindent Load in the data labeled \texttt{climateSupport.csv} on GitHub, which contains an observational study of 8,500 observations.

  \begin{lstlisting}[language=R]
    load(url("https://github.com/ASDS-TCD/StatsII_Spring2023/blob/main/datasets/climateSupport.RData?raw=true"))

    # choice = 1,2
    # countries = 1, 2, 3
    # sanctions = 1, 2, 3, 4

    csFacs <- climateSupport
    csFacs$choice <- relevel(csFacs$choice, "Supported")
  \end{lstlisting}

\begin{itemize}
	\item
	Response variable: 
	\begin{itemize}
		\item \texttt{choice}: 1 if the individual agreed with the policy; 0 if the individual did not support the policy
	\end{itemize}
	\item
	Explanatory variables: 
	\begin{itemize}
		\item
		\texttt{countries}: Number of participating countries [20 of 192; 80 of 192; 160 of 192]
		\item
		\texttt{sanctions}: Sanctions for missing emission reduction targets [None, 5\%, 15\%, and 20\% of the monthly household costs given 2\% GDP growth]
		
	\end{itemize}
	
\end{itemize}

%\newpage
\noindent Please answer the following questions:

\begin{enumerate}
	\item
	Remember, we are interested in predicting the likelihood of an individual supporting a policy based on the number of countries participating and the possible sanctions for non-compliance.
	\begin{enumerate}
		\item [] Fit an additive model. 
      \lstinputlisting[language=R, firstline=95, lastline=95]{PS2_ImeldaFinn.R}
		
		\item Provide the summary output, 

    \begin{lstlisting}
    Call:
    glm(formula = choice ~ ., family = binomial(link = "logit"), data = csFacs)

    Deviance Residuals: 
        Min       1Q   Median       3Q      Max  
    -1.4259  -1.1480  -0.9444   1.1505   1.4298  

    Coefficients:
                 Estimate Std. Error z value Pr(>|z|)    
    (Intercept) -0.005665   0.021971  -0.258 0.796517    
    countries.L  0.458452   0.038101  12.033  < 2e-16 ***
    countries.Q -0.009950   0.038056  -0.261 0.793741    
    sanctions.L -0.276332   0.043925  -6.291 3.15e-10 ***
    sanctions.Q -0.181086   0.043963  -4.119 3.80e-05 ***
    sanctions.C  0.150207   0.043992   3.414 0.000639 ***
    ---
    Signif. codes:  0 '***' 0.001 '**' 0.01 '*' 0.05 '.' 0.1 ' ' 1

    (Dispersion parameter for binomial family taken to be 1)

    Null deviance: 11783  on 8499  degrees of freedom
    Residual deviance: 11568  on 8494  degrees of freedom
    AIC: 11580

    Number of Fisher Scoring iterations: 4

    \end{lstlisting}
		
		\item the global null hypothesis:
		
		$H_0$: the explanatory variables have no effect on the likelihood of an individual supporting a policy
		
		$H_a$: one or more of the explanatory variables have some effect on the likelihood of an individual supporting a policy

    $\alpha = 0.05$
    
		\item and $p$-value. 
      The model was run with no explanatory variables (dummy... TODO).  The comparison of the two models is shown in ~\ref{tab:glm:null}

%	    
% Table created by stargazer v.5.2.3 by Marek Hlavac, Social Policy Institute. E-mail: marek.hlavac at gmail.com
% Date and time: Thu, Feb 16, 2023 - 21:02:04
\begin{table}[!htbp] \centering 
  \caption{} 
  \label{tab:glm:null} 
\begin{tabular}{@{\extracolsep{5pt}}lcc} 
\\[-1.8ex]\hline 
\hline \\[-1.8ex] 
 & \multicolumn{2}{c}{\textit{Dependent variable:}} \\ 
\cline{2-3} 
\\[-1.8ex] & \multicolumn{2}{c}{choice} \\ 
\\[-1.8ex] & (1) & (2)\\ 
\hline \\[-1.8ex] 
 countries.L & $-$0.458$^{***}$ &  \\ 
  & (0.038) &  \\ 
  & & \\ 
 countries.Q & 0.010 &  \\ 
  & (0.038) &  \\ 
  & & \\ 
 sanctions.L & 0.276$^{***}$ &  \\ 
  & (0.044) &  \\ 
  & & \\ 
 sanctions.Q & 0.181$^{***}$ &  \\ 
  & (0.044) &  \\ 
  & & \\ 
 sanctions.C & $-$0.150$^{***}$ &  \\ 
  & (0.044) &  \\ 
  & & \\ 
 Constant & 0.006 & 0.007 \\ 
  & (0.022) & (0.022) \\ 
  & & \\ 
\hline \\[-1.8ex] 
Observations & 8,500 & 8,500 \\ 
Log Likelihood & $-$5,784.130 & $-$5,891.705 \\ 
Akaike Inf. Crit. & 11,580.260 & 11,785.410 \\ 
\hline 
\hline \\[-1.8ex] 
\textit{Note:}  & \multicolumn{2}{r}{$^{*}$p$<$0.1; $^{**}$p$<$0.05; $^{***}$p$<$0.01} \\ 
\end{tabular} 
\end{table}  

%	    
% Table created by stargazer v.5.2.3 by Marek Hlavac, Social Policy Institute. E-mail: marek.hlavac at gmail.com
% Date and time: Thu, Feb 16, 2023 - 21:02:17
\begin{table}[!htbp] \centering 
  \caption{} 
  \label{tab:anova} 
\begin{tabular}{@{\extracolsep{5pt}}lccccc} 
\\[-1.8ex]\hline 
\hline \\[-1.8ex] 
Statistic & \multicolumn{1}{c}{N} & \multicolumn{1}{c}{Mean} & \multicolumn{1}{c}{St. Dev.} & \multicolumn{1}{c}{Min} & \multicolumn{1}{c}{Max} \\ 
\hline \\[-1.8ex] 
Resid. Df & 2 & 8,496.500 & 3.536 & 8,494 & 8,499 \\ 
Resid. Dev & 2 & 11,675.830 & 152.134 & 11,568.260 & 11,783.410 \\ 
Df & 1 & 5.000 &  & 5 & 5 \\ 
Deviance & 1 & 215.150 &  & 215.150 & 215.150 \\ 
\hline \\[-1.8ex] 
\end{tabular} 
\end{table}  

			
% Table created by stargazer v.5.2.3 by Marek Hlavac, Social Policy Institute. E-mail: marek.hlavac at gmail.com
% Date and time: Fri, Feb 17, 2023 - 23:19:51
\begin{table}[!htbp] \centering 
  \caption{} 
  \label{tab:glm:null} 
\begin{tabular}{@{\extracolsep{5pt}}lcc} 
\\[-1.8ex]\hline 
\hline \\[-1.8ex] 
 & \multicolumn{2}{c}{\textit{Dependent variable:}} \\ 
\cline{2-3} 
\\[-1.8ex] & \multicolumn{2}{c}{choice} \\ 
\\[-1.8ex] & \multicolumn{2}{c}{\textit{logistic}} \\ 
\\[-1.8ex] & (1) & (2)\\ 
\hline \\[-1.8ex] 
 countries:  80 of 192 & 0.458$^{***}$ &  \\ 
  & (0.038) &  \\ 
  & & \\ 
 countries:  160 of 192 & $-$0.010 &  \\ 
  & (0.038) &  \\ 
  & & \\ 
 sanctions: 5\% & $-$0.276$^{***}$ &  \\ 
  & (0.044) &  \\ 
  & & \\ 
 sanctions: 5\% & $-$0.181$^{***}$ &  \\ 
  & (0.044) &  \\ 
  & & \\ 
 sanctions: 5\% & 0.150$^{***}$ &  \\ 
  & (0.044) &  \\ 
  & & \\ 
 Constant & $-$0.006 & $-$0.007 \\ 
  & (0.022) & (0.022) \\ 
  & & \\ 
\hline \\[-1.8ex] 
Observations & 8,500 & 8,500 \\ 
Log Likelihood & $-$5,784.130 & $-$5,891.705 \\ 
Akaike Inf. Crit. & 11,580.260 & 11,785.410 \\ 
\hline 
\hline \\[-1.8ex] 
\textit{Note:}  & \multicolumn{2}{r}{$^{*}$p$<$0.1; $^{**}$p$<$0.05; $^{***}$p$<$0.01} \\ 
\end{tabular} 
\end{table}  

% Table created by stargazer v.5.2.3 by Marek Hlavac, Social Policy Institute. E-mail: marek.hlavac at gmail.com
% Date and time: Fri, Feb 17, 2023 - 23:19:55
\begin{table}[!htbp] \centering 
  \caption{} 
  \label{tab:anova} 
\begin{tabular}{@{\extracolsep{5pt}}lccccc} 
\\[-1.8ex]\hline 
\hline \\[-1.8ex] 
Statistic & \multicolumn{1}{c}{N} & \multicolumn{1}{c}{Mean} & \multicolumn{1}{c}{St. Dev.} & \multicolumn{1}{c}{Min} & \multicolumn{1}{c}{Max} \\ 
\hline \\[-1.8ex] 
Resid. Df & 2 & 8,496.500 & 3.536 & 8,494 & 8,499 \\ 
Resid. Dev & 2 & 11,675.830 & 152.134 & 11,568.260 & 11,783.410 \\ 
Df & 1 & 5.000 &  & 5 & 5 \\ 
Deviance & 1 & 215.150 &  & 215.150 & 215.150 \\ 
Pr(\textgreater Chi) & 1 & 0.000 &  & 0 & 0 \\ 
\hline \\[-1.8ex] 
\end{tabular} 
\end{table}  


		\item  Please describe the results and provide a conclusion.

      Table~\ref{tab:CIs}
      
      The $e^{\beta_k}$s are all non-zero and the $5\%$ Confidence Intervals do not include 0.(Table~\ref{tab:CIs}).
      
      The estimates for $\beta_k$ are all significant at $p=0.01$ except for \texttt{countries.Q}, ie there is no significant difference in likelihood going from 80 to 160 countries.
      
	    
% Table created by stargazer v.5.2.3 by Marek Hlavac, Social Policy Institute. E-mail: marek.hlavac at gmail.com
% Date and time: Fri, Feb 17, 2023 - 23:20:37
\begin{table}[!htbp] \centering 
  \caption{} 
  \label{tab:CIs} 
\begin{tabular}{@{\extracolsep{5pt}} cccc} 
\\[-1.8ex]\hline 
\hline \\[-1.8ex] 
 & lower & coefs & upper \\ 
\hline \\[-1.8ex] 
(Intercept) & $0.952$ & $0.994$ & $1.038$ \\ 
countries.L & $1.468$ & $1.582$ & $1.704$ \\ 
countries.Q & $0.919$ & $0.990$ & $1.067$ \\ 
sanctions.L & $0.696$ & $0.759$ & $0.827$ \\ 
sanctions.Q & $0.765$ & $0.834$ & $0.909$ \\ 
sanctions.C & $1.066$ & $1.162$ & $1.267$ \\ 
\hline \\[-1.8ex] 
\end{tabular} 
\end{table}  


		It took 4 iterations to find the maximum likelihood estimates.
		
		the log likelihood is -5,784.130
	\end{enumerate}

%\begin{verbatim}

%\end{verbatim}
	

	\item
	If any of the explanatory variables are significant in this model, then:
	\begin{enumerate}
		\item
		For the policy in which nearly all countries participate [160 of 192], how does increasing sanctions from 5\% to 15\% change the odds that an individual will support the policy? (Interpretation of a coefficient)
%		\item
%		For the policy in which very few countries participate [20 of 192], how does increasing sanctions from 5\% to 15\% change the odds that an individual will support the policy? (Interpretation of a coefficient)
		\item
		What is the estimated probability that an individual will support a policy if there are 80 of 192 countries participating with no sanctions? 
		\item
		Would the answers to 2a and 2b potentially change if we included the interaction term in this model? Why? 
		\begin{itemize}
			\item Perform a test to see if including an interaction is appropriate.
			
			  
% Table created by stargazer v.5.2.3 by Marek Hlavac, Social Policy Institute. E-mail: marek.hlavac at gmail.com
% Date and time: Thu, Feb 16, 2023 - 21:07:13
\begin{table}[!htbp] \centering 
  \caption{} 
  \label{tab:glm:int} 
\begin{tabular}{@{\extracolsep{5pt}}lccc} 
\\[-1.8ex]\hline 
\hline \\[-1.8ex] 
 & \multicolumn{3}{c}{\textit{Dependent variable:}} \\ 
\cline{2-4} 
\\[-1.8ex] & \multicolumn{3}{c}{choice} \\ 
\\[-1.8ex] & (1) & (2) & (3)\\ 
\hline \\[-1.8ex] 
 countries.L & $-$0.458$^{***}$ &  & $-$0.457$^{***}$ \\ 
  & (0.038) &  & (0.038) \\ 
  & & & \\ 
 countries.Q & 0.010 &  & 0.011 \\ 
  & (0.038) &  & (0.038) \\ 
  & & & \\ 
 sanctions.L & 0.276$^{***}$ &  & 0.274$^{***}$ \\ 
  & (0.044) &  & (0.044) \\ 
  & & & \\ 
 sanctions.Q & 0.181$^{***}$ &  & 0.182$^{***}$ \\ 
  & (0.044) &  & (0.044) \\ 
  & & & \\ 
 sanctions.C & $-$0.150$^{***}$ &  & $-$0.153$^{***}$ \\ 
  & (0.044) &  & (0.044) \\ 
  & & & \\ 
 countries.L:sanctions.L &  &  & 0.002 \\ 
  &  &  & (0.077) \\ 
  & & & \\ 
 countries.Q:sanctions.L &  &  & $-$0.134$^{*}$ \\ 
  &  &  & (0.076) \\ 
  & & & \\ 
 countries.L:sanctions.Q &  &  & 0.008 \\ 
  &  &  & (0.076) \\ 
  & & & \\ 
 countries.Q:sanctions.Q &  &  & $-$0.093 \\ 
  &  &  & (0.076) \\ 
  & & & \\ 
 countries.L:sanctions.C &  &  & $-$0.095 \\ 
  &  &  & (0.076) \\ 
  & & & \\ 
 countries.Q:sanctions.C &  &  & $-$0.010 \\ 
  &  &  & (0.077) \\ 
  & & & \\ 
 Constant & 0.006 & 0.007 & 0.004 \\ 
  & (0.022) & (0.022) & (0.022) \\ 
  & & & \\ 
\hline \\[-1.8ex] 
Observations & 8,500 & 8,500 & 8,500 \\ 
Log Likelihood & $-$5,784.130 & $-$5,891.705 & $-$5,780.983 \\ 
Akaike Inf. Crit. & 11,580.260 & 11,785.410 & 11,585.970 \\ 
\hline 
\hline \\[-1.8ex] 
\textit{Note:}  & \multicolumn{3}{r}{$^{*}$p$<$0.1; $^{**}$p$<$0.05; $^{***}$p$<$0.01} \\ 
\end{tabular} 
\end{table}  

    	  
% Table created by stargazer v.5.2.3 by Marek Hlavac, Social Policy Institute. E-mail: marek.hlavac at gmail.com
% Date and time: Thu, Feb 16, 2023 - 21:07:15
\begin{table}[!htbp] \centering 
  \caption{ANOVA} 
  \label{tab:anovas} 
\begin{tabular}{@{\extracolsep{5pt}}lccccc} 
\\[-1.8ex]\hline 
\hline \\[-1.8ex] 
Statistic & \multicolumn{1}{c}{N} & \multicolumn{1}{c}{Mean} & \multicolumn{1}{c}{St. Dev.} & \multicolumn{1}{c}{Min} & \multicolumn{1}{c}{Max} \\ 
\hline \\[-1.8ex] 
Resid. Df & 2 & 8,496.500 & 3.536 & 8,494 & 8,499 \\ 
Resid. Dev & 2 & 11,675.830 & 152.134 & 11,568.260 & 11,783.410 \\ 
Df & 1 & 5.000 &  & 5 & 5 \\ 
Deviance & 1 & 215.150 &  & 215.150 & 215.150 \\ 
\hline \\[-1.8ex] 
\end{tabular} 
\end{table} 

% Table created by stargazer v.5.2.3 by Marek Hlavac, Social Policy Institute. E-mail: marek.hlavac at gmail.com
% Date and time: Thu, Feb 16, 2023 - 21:07:15
\begin{table}[!htbp] \centering 
  \caption{ANOVA} 
  \label{tab:anovas} 
\begin{tabular}{@{\extracolsep{5pt}}lccccc} 
\\[-1.8ex]\hline 
\hline \\[-1.8ex] 
Statistic & \multicolumn{1}{c}{N} & \multicolumn{1}{c}{Mean} & \multicolumn{1}{c}{St. Dev.} & \multicolumn{1}{c}{Min} & \multicolumn{1}{c}{Max} \\ 
\hline \\[-1.8ex] 
Resid. Df & 2 & 8,491.000 & 4.243 & 8,488 & 8,494 \\ 
Resid. Dev & 2 & 11,565.110 & 4.450 & 11,561.970 & 11,568.260 \\ 
Df & 1 & 6.000 &  & 6 & 6 \\ 
Deviance & 1 & 6.293 &  & 6.293 & 6.293 \\ 
\hline \\[-1.8ex] 
\end{tabular} 
\end{table} 

% Table created by stargazer v.5.2.3 by Marek Hlavac, Social Policy Institute. E-mail: marek.hlavac at gmail.com
% Date and time: Thu, Feb 16, 2023 - 21:07:15
\begin{table}[!htbp] \centering 
  \caption{ANOVA} 
  \label{tab:anovas} 
\begin{tabular}{@{\extracolsep{5pt}} c} 
\\[-1.8ex]\hline 
\hline \\[-1.8ex] 
ANOVA \\ 
\hline \\[-1.8ex] 
\end{tabular} 
\end{table}  

    	  %
% Table created by stargazer v.5.2.3 by Marek Hlavac, Social Policy Institute. E-mail: marek.hlavac at gmail.com
% Date and time: Thu, Feb 16, 2023 - 21:07:18
\begin{table}[!htbp] \centering 
  \caption{} 
  \label{tab:all} 
\begin{tabular}{@{\extracolsep{5pt}}lccc} 
\\[-1.8ex]\hline 
\hline \\[-1.8ex] 
 & \multicolumn{3}{c}{\textit{Dependent variable:}} \\ 
\cline{2-4} 
\\[-1.8ex] & \multicolumn{3}{c}{choice} \\ 
\\[-1.8ex] & (1) & (2) & (3)\\ 
\hline \\[-1.8ex] 
 countries.L & $-$0.458$^{***}$ &  & $-$0.457$^{***}$ \\ 
  & (0.038) &  & (0.038) \\ 
  & & & \\ 
 countries.Q & 0.010 &  & 0.011 \\ 
  & (0.038) &  & (0.038) \\ 
  & & & \\ 
 sanctions.L & 0.276$^{***}$ &  & 0.274$^{***}$ \\ 
  & (0.044) &  & (0.044) \\ 
  & & & \\ 
 sanctions.Q & 0.181$^{***}$ &  & 0.182$^{***}$ \\ 
  & (0.044) &  & (0.044) \\ 
  & & & \\ 
 sanctions.C & $-$0.150$^{***}$ &  & $-$0.153$^{***}$ \\ 
  & (0.044) &  & (0.044) \\ 
  & & & \\ 
 countries.L:sanctions.L &  &  & 0.002 \\ 
  &  &  & (0.077) \\ 
  & & & \\ 
 countries.Q:sanctions.L &  &  & $-$0.134$^{*}$ \\ 
  &  &  & (0.076) \\ 
  & & & \\ 
 countries.L:sanctions.Q &  &  & 0.008 \\ 
  &  &  & (0.076) \\ 
  & & & \\ 
 countries.Q:sanctions.Q &  &  & $-$0.093 \\ 
  &  &  & (0.076) \\ 
  & & & \\ 
 countries.L:sanctions.C &  &  & $-$0.095 \\ 
  &  &  & (0.076) \\ 
  & & & \\ 
 countries.Q:sanctions.C &  &  & $-$0.010 \\ 
  &  &  & (0.077) \\ 
  & & & \\ 
 Constant & 0.006 & 0.007 & 0.004 \\ 
  & (0.022) & (0.022) & (0.022) \\ 
  & & & \\ 
\hline \\[-1.8ex] 
Observations & 8,500 & 8,500 & 8,500 \\ 
Log Likelihood & $-$5,784.130 & $-$5,891.705 & $-$5,780.983 \\ 
Akaike Inf. Crit. & 11,580.260 & 11,785.410 & 11,585.970 \\ 
\hline 
\hline \\[-1.8ex] 
\textit{Note:}  & \multicolumn{3}{r}{$^{*}$p$<$0.1; $^{**}$p$<$0.05; $^{***}$p$<$0.01} \\ 
\end{tabular} 
\end{table} 

% Table created by stargazer v.5.2.3 by Marek Hlavac, Social Policy Institute. E-mail: marek.hlavac at gmail.com
% Date and time: Thu, Feb 16, 2023 - 21:07:19
\begin{table}[!htbp] \centering 
  \caption{} 
  \label{tab:all} 
\begin{tabular}{@{\extracolsep{5pt}}lccccc} 
\\[-1.8ex]\hline 
\hline \\[-1.8ex] 
Statistic & \multicolumn{1}{c}{N} & \multicolumn{1}{c}{Mean} & \multicolumn{1}{c}{St. Dev.} & \multicolumn{1}{c}{Min} & \multicolumn{1}{c}{Max} \\ 
\hline \\[-1.8ex] 
Resid. Df & 2 & 8,496.500 & 3.536 & 8,494 & 8,499 \\ 
Resid. Dev & 2 & 11,675.830 & 152.134 & 11,568.260 & 11,783.410 \\ 
Df & 1 & 5.000 &  & 5 & 5 \\ 
Deviance & 1 & 215.150 &  & 215.150 & 215.150 \\ 
\hline \\[-1.8ex] 
\end{tabular} 
\end{table} 

% Table created by stargazer v.5.2.3 by Marek Hlavac, Social Policy Institute. E-mail: marek.hlavac at gmail.com
% Date and time: Thu, Feb 16, 2023 - 21:07:19
\begin{table}[!htbp] \centering 
  \caption{} 
  \label{tab:all} 
\begin{tabular}{@{\extracolsep{5pt}}lccccc} 
\\[-1.8ex]\hline 
\hline \\[-1.8ex] 
Statistic & \multicolumn{1}{c}{N} & \multicolumn{1}{c}{Mean} & \multicolumn{1}{c}{St. Dev.} & \multicolumn{1}{c}{Min} & \multicolumn{1}{c}{Max} \\ 
\hline \\[-1.8ex] 
Resid. Df & 2 & 8,491.000 & 4.243 & 8,488 & 8,494 \\ 
Resid. Dev & 2 & 11,565.110 & 4.450 & 11,561.970 & 11,568.260 \\ 
Df & 1 & 6.000 &  & 6 & 6 \\ 
Deviance & 1 & 6.293 &  & 6.293 & 6.293 \\ 
\hline \\[-1.8ex] 
\end{tabular} 
\end{table}  


		\end{itemize}
	\end{enumerate}
	\end{enumerate}


\end{document}
