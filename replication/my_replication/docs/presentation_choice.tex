\documentclass{beamer}
% There are many different themes available for Beamer. A comprehensive
% list with examples is given here:
% http://deic.uab.es/~iblanes/beamer_gallery/index_by_theme.html
% You can uncomment the themes below if you would like to use a different
% one:
\usetheme{Montpellier}
%\usetheme{Antibes}
%\usetheme{Berkeley} %!
%\usetheme{Berlin}
%\usetheme{Boadilla} %!
%\usetheme{Frankfurt}
%\usetheme{boxes}   %!
%\usetheme{CambridgeUS}
%\usetheme{Copenhagen}
%\usetheme{Darmstadt} %!
%\usepackage{hologo}

\usepackage{tikz}
\usepackage{url}
\usepackage{color}
\usepackage{longtable}
\usepackage{booktabs}
\usetikzlibrary{shapes.geometric, arrows, positioning, calc}
\tikzstyle{startStop} = [rectangle, rounded corners, minimum width=7cm, text width=11cm, text centered, minimum height=.5cm, draw=black]
\tikzstyle{io} = [circle, rounded corners, minimum width=1cm, text width=1.5cm, minimum height=.1, text centered, draw=black]
\tikzstyle{arrow} = [thick,->,>=stealth]
\usepackage{pdfpages}
\usepackage{graphicx}   % need for figures
\usepackage{adjustbox}
\usepackage{fontawesome}
\usepackage[absolute,overlay]{textpos}
%CHANGES COLOR TO GREEN
\usepackage{listings}
\usepackage{color}
% for game design diagram
\usetikzlibrary{trees}

\usepackage{verbatim}

\definecolor{codegreen}{rgb}{0,0.6,0}
\definecolor{codegray}{rgb}{0.5,0.5,0.5}
\definecolor{codepurple}{rgb}{0.58,0,0.82}
\definecolor{backcolour}{rgb}{0.95,0.95,0.92}

\lstdefinestyle{mystyle}{
	backgroundcolor=\color{backcolour},
	commentstyle=\color{codegreen},
	keywordstyle=\color{magenta},
	numberstyle=\tiny\color{codegray},
	stringstyle=\color{codepurple},
	basicstyle=\scriptsize,
	breakatwhitespace=true,
	breaklines=true,
	captionpos=b,
	keepspaces=false,
	numbers=left,
	numbersep=5pt,
	showspaces=false,
	showstringspaces=false,
	showtabs=false,
	tabsize=2
}
\hypersetup{
	colorlinks = true,
	linkcolor=blue,   % color of internal links
	citecolor=blue,   % color of links to bibliography
	urlcolor=blue,    % color of external links
	%	pagebackref=true,
	%	implicit=false,
	%	bookmarks=true,
	bookmarksopen=true,
	pdfdisplaydoctitle=true
}
\lstset{style=mystyle}
\definecolor{brickred}{rgb}{0.8, 0.25, 0.33}
\definecolor{mygreen}{cmyk}{0.82,0.11,1,0.25}

\setbeamertemplate{blocks}[rounded][shadow=false]
\addtobeamertemplate{block begin}{\pgfsetfillopacity{0.8}}{\pgfsetfillopacity{1}}
\setbeamertemplate{background canvas}{ \begin{tikzpicture}
	\node[opacity=.05]{
		\includegraphics [width=\paperwidth]{../imgs/greenyellow.jpg}};
\end{tikzpicture}
}
\setbeamertemplate{caption}{}
%\setbeamercolor{structure}{fg=mygreen}
%\setbeamercolor*{block title example}{fg=white,bg= white}
%\setbeamercolor*{block body example}{fg= white,bg= white}
\usepackage[english]{babel}
\usepackage{hyperref}
\usepackage{dcolumn}
\usepackage{adjustbox}
\usepackage{multicol}
\usepackage{adjustbox}
\usepackage{amsmath}
\usepackage{tikz}
\usepackage[all,cmtip]{xy}
\tikzstyle{largeSquare} = [rectangle, rounded corners, minimum width=7cm, text width=9cm, minimum height=.5cm, draw=black]
\usepackage{tikzsymbols}

\usetikzlibrary{shapes.geometric, arrows}
\tikzstyle{arrow}=[thick,->,>=stealth]
\beamertemplatenavigationsymbolsempty
\usepackage{subfloat}
\setbeamertemplate{headline}{}
\newcommand{\speechthis}[2]{
	\tikz[remember picture,baseline]{\node[anchor=base,inner sep=0,outer sep=0]
		(#1) {\underline{#1}};\node[overlay,ellipse callout,fill=blue!50]
		at ($(#1.north)+(-.5cm,0.8cm)$) {#2};}
}

%\setbeamercolor{button}{bg=mygreen,fg=white}
%\usecolortheme{beaver} %TODO

\setbeamercovered{invisible}


\definecolor{codegreen}{rgb}{0,0.6,0}
\definecolor{codegray}{rgb}{0.5,0.5,0.5}
\definecolor{codepurple}{rgb}{0.58,0,0.82}
\definecolor{backcolour}{rgb}{0.95,0.95,0.92}

\lstdefinestyle{mystyle}{
	backgroundcolor=\color{backcolour},
	commentstyle=\color{codegreen},
	keywordstyle=\color{magenta},
	numberstyle=\tiny\color{codegray},
	stringstyle=\color{codepurple},
	basicstyle=\footnotesize,
	breakatwhitespace=false,
	breaklines=true,
	captionpos=b,
	keepspaces=true,
	numbers=left,
	numbersep=5pt,
	showspaces=false,
	showstringspaces=false,
	showtabs=false,
	tabsize=2
}
\lstset{style=mystyle}
\newcommand{\Sref}[1]{Section~\ref{#1}}
\newtheorem{hyp}{Hypothesis}

\title{Choice and Personal responsibility\\What is a morally relevant choice?}
\author{Imelda Finn}
\subtitle{Applied Statistical Analysis II}
\date{Spring 2023}
\begin{document}
	\frame{\titlepage}

	\begin{frame}{Background/introduction}

		\begin{block}

			\begin{itemize}

				\item Introduction \vspace{.25cm}
				\begin{itemize}
					\item Alexander W. Cappelen, Sebastian Fest, Erik Ø. Sørensen, Bertil Tungodden
					\item September 12, 2018
					\item Norwegian School of Economics, Bergen, Norway
					\item \href{https://dataverse.harvard.edu/dataset.xhtml?persistentId=doi:10.7910/DVN/A6KFNO}{Data from Harvard Dataverse, FAIR - Centre for Experimental Research on Fairness, Inequality and Rationality}
					\item 					\url{https://cee.boun.edu.tr/sites/cee.boun.edu.tr/files/documents/CEE2018Conference/cprwmrc.pdf} %cite
				\end{itemize}
			\end{itemize}
		\end{block}
	\end{frame}

	\begin{frame}{Theory}

	\begin{block}

		\begin{itemize}
			\item Inequality is tolerated because people are blamed for their choices/outcomes.
			\item A person should \textbf{not} be held personally responsible for the outcome of a choice if:
			\begin{itemize}
				\item the person could not have changed the likelihood of the outcome by choosing differently (\textbf{no ex ante causal responsibility}), or
				\item  the person could only have avoided the outcome at unreasonably large cost (\textbf{no acceptable alternative}).
			\end{itemize}
			\item 	$H_0$ a trivial choice will make no difference to the outcome.
		\end{itemize}
	\end{block}
\end{frame}

%------------------------------------------------------------------------------------
	\begin{frame}{Game Design}
		%\documentclass{article}

%\usepackage[latin1]{inputenc}
%\usepackage{tikz}
%\usetikzlibrary{trees}
%\begin{document}
%\thispagestyle{empty}

% Set the overall layout of the tree
\tikzstyle{level 1}=[level distance=3.0cm, sibling distance=5.5cm]
\tikzstyle{level 2}=[level distance=3.5cm, sibling distance=3.0cm]
\tikzstyle{level 3}=[level distance=3.5cm, sibling distance=2.0cm]

% Define styles for bags and leafs
\tikzstyle{bag} = [text width=4em, text centered]
\tikzstyle{end} = [circle, minimum width=3pt,fill, inner sep=0pt]

% The sloped option gives rotated edge labels. Personally
% I find sloped labels a bit difficult to read. Remove the sloped options
% to get horizontal labels. 
\begin{figure}
\begin{tikzpicture}[grow=right, sloped]
\node[bag] {Game Tree}
  child{
    node[bag] {Base }
    child {
        node[bag] {Nature}        
            child {
                node[end, label=right:{L}] {}
                edge from parent
                node[above] {yellow}
                node[below] {$p=0.5$}
            }
            child {
                node[end, label=right:{0}] {}
                edge from parent
                node[above] {green}
                node[below] {$p=0.5$}
            }
            edge from parent 
%            node[above] {lottery}
    }
  }
  child{
    node[bag] {Nominal Choice \\Participant}
    child {
        node[bag] {Nature}        
            child {
                node[end, label=right:{L}] {}
                edge from parent
                node[above] {yellow}
                node[below] {$p=0.5$}
            }
            child {
                node[end, label=right:{0}] {}
                edge from parent
                node[above] {green}
                node[below] {$p=0.5$}
            }
            edge from parent 
            node[above] {yellow}
    }
    child {
        node[bag] {Nature}        
            child {
                node[end, label=right:{L}] {}
                edge from parent
                node[above] {yellow}
                node[below] {$p=0.5$}
            }
            child {
                node[end, label=right:{0}] {}
                edge from parent
                node[above] {green}
                node[below] {$p=0.5$}
            }
            edge from parent 
            node[above] {green}
    }
  }
  child{
    node[bag] {Forced Choice \\Participant}
    child {
        node[bag] {Nature}        
            child {
                node[end, label=above:{L}] {}
                edge from parent
                node[above] {yellow}
                node[below] {$p=0.5$}
            }
            child {
                node[end, label=above:{0}] {}
                edge from parent
                node[above] {green}
                node[below] {$p=0.5$}
            }
            edge from parent 
            node[above] {lottery}
    }
    child {
        node[end,label=above:{S}] {}      
            edge from parent 
            node[above] {safe}
    }
  }
  ;
\end{tikzpicture}
\caption{Note: The figure shows the sequential form game representation of how the earnings were determined in each of the three treatments
in the experiments. In the lab experiment, L = 800 and S = 25 (in NOK). In the online experiment, L = 8 and $S\in (-0.25, 0, 0.25)$ (in USD).}
\label{fig:gametree}
\end{figure}

%\end{document}
	\end{frame}
%------------------------------------------------------------------------------------

	\begin{frame}{Experimental Design}

	\begin{block}

		\begin{columns}[T]
		\begin{column}{0.23\linewidth}
		~\includegraphics[height=0.3\textheight,keepaspectratio]{../imgs/gumball.jpg}
		\end{column}
		\begin{column}{0.75\linewidth}
		\begin{itemize}
	\item Studying spectators' decision to redistribute money/not
	\item payment allocated in all cases by lottery -  green/yellow ball.
	\item random allocation to treatment, no interaction - blind trial
	\item background information about age, gender, and political orientation, education, income.
	\item three-item cognitive reflection test measuring the ability to correct for incorrect
	intuitive answers through reflection.
\end{itemize}
		\end{column}
		\end{columns}
		
	\end{block}
\end{frame}

%------------------------------------------------------------------------------------
\begin{frame}[fragile]{Experimental Design - Variables}

	\begin{block}{Dependent Variables}\small
		\begin{itemize}
			\item \texttt{inequality} inequality implemented by the spectator
			\[ inequality = \frac{|Income Lucky - Income Unlucky|}{Total Income} \in [0,1]\]

			Coded as:
			\begin{lstlisting}
				abs(800 - 2*transfer)/800
				abs(8 - 2*y)/8.0
			\end{lstlisting}

			\begin{itemize}
			\item no transfer \texttt{inequality} =1;
			\item equal split: \texttt{inequality} =0; \item full transfer \texttt{inequality} =1
			\end{itemize}			\item \texttt{zero\_to\_worst\_off} : 1 if the spectator does not assign any income to one of the participants

			\begin{lstlisting}
				(y %in% c(0,8))
			\end{lstlisting}
		\end{itemize}
	\end{block}


\end{frame}


%------------------------------------------------------------------------------------
	\begin{frame}{Lab Experiment}

	\begin{block}

		\begin{itemize}
			\item 422 participants from 2 Norwegian colleges
			\item average age 22.7 years, 54\% male, average CRT score 1.6/3, 41\% self-reported support for a right-wing party in Norway\footnote{close to the distribution of votes in the last election in Norway.}
			\item 800 NOK or 0
			\item all worked at simple task
			\item  spectators were participants; they didn't know when making choice if they had gotten money
			\item spectators asked what motivated their decision to redistribute/not.
			\item Average payment was 475 NOK (approximately 80 USD) incl 100 basic fee.
		\end{itemize}
	\end{block}
\end{frame}
%------------------------------------------------------------------------------------
\begin{frame}{Experimental Design - Lab - Variables}
	
	\begin{block}{Treatment variables}\small
		
		\begin{itemize}
			\item \texttt{treatment} $\in$ (``Base'', “Forced Choice”, “Nominal Choice”)
			\item \texttt{choice} TRUE if \texttt{treatment} $\in$ (“Forced Choice”, “Nominal Choice”)
		\end{itemize}
	\end{block}
	
	\begin{block}{Spectator Variables}\small
		\begin{itemize}
			\item \texttt{leftp}: spectator self-reporting that he or she voted for a non-right-wing party in the last election
			\item \texttt{female}: TRUE if female
			\item \texttt{age\_h}: TRUE if age $\ge$ median
			\item \texttt{crt\_h}: TRUE if spectator's score on cognitive reflection test $\ge$ median (2 out of 3).
		\end{itemize}
	\end{block}
	
\end{frame}

%-----------------------------------------------------------------------------------
  	\begin{frame}{Lab Experiment Transfers}
		\includegraphics[height=.7\textheight]{../graphs/histograms_lab.pdf}

		{\tiny
			\emph{Note}: The figure shows the histogram of the amount of money transferred from the lucky to the unlucky participant by the spectator in each treatment.}
	\end{frame}
%------------------------------------------------------------------------------------
\begin{frame}{Lab Experiment Inequality}
	\includegraphics[height=.75\textheight]{../graphs/mean_ineq_nothing_lab.pdf}

		{\tiny
		\emph{Note}: The left panel shows the average inequality implemented by the spectators in each
		treatment, the right panel shows the share of spectators assigning no income to one of the
		participants in the pair in each of the treatments. The standard errors of the mean are indicated.}
\end{frame}

%------------------------------------------------------------------------------------
\begin{frame}{Table 1: Lab Results Regression 1: the role of choice}
		
		\begin{block}\tiny

\begin{table}[!htbp] \centering \tiny
	\caption{} 
	\label{tbl:l1} 
	\begin{tabular}{@{\extracolsep{4pt}}lcccc} 
		\\[-1.0ex]\hline 
		\hline \\[-1.2ex] 
		\\[-1.0ex] & \multicolumn{2}{c}{inequality} & \multicolumn{2}{c}{zero\_to\_worst\_off} \\ 
		\\[-1.0ex] & (1) & (2) & (3) & (4)\\ 
		\hline \\[-1.0ex] 
		treatmentForced Choice & 0.120 & 0.125 & 0.094 & 0.101 \\ 
		& (0.044) & (0.044) & (0.043) & (0.042) \\ 
		& p = 0.007 & p = 0.005 & p = 0.028 & p = 0.017 \\ 
%		& & & & \\ 
		treatmentNominal Choice & 0.164 & 0.163 & 0.125 & 0.128 \\ 
		& (0.044) & (0.044) & (0.044) & (0.043) \\ 
		& p = 0.001 & p = 0.001 & p = 0.005 & p = 0.003 \\ 
%		& & & & \\ 
		leftp &  & $-$0.115 &  & $-$0.075 \\ 
		&  & (0.037) &  & (0.037) \\ 
		&  & p = 0.003 &  & p = 0.044 \\ 
%		& & & & \\ 
		female &  & $-$0.108 &  & $-$0.159 \\ 
		&  & (0.040) &  & (0.039) \\ 
		&  & p = 0.007 &  & p = 0.00 \\ 
%		& & & & \\ 
		age\_h &  & 0.017 &  & 0.051 \\ 
		&  & (0.037) &  & (0.036) \\ 
		&  & p = 0.646 &  & p = 0.157 \\ 
%		& & & & \\ 
		crt\_h &  & 0.001 &  & 0.009 \\ 
		&  & (0.040) &  & (0.039) \\ 
		&  & p = 0.984 &  & p = 0.827 \\ 
%		& & & & \\ 
		Constant & 0.204 & 0.310 & 0.103 & 0.182 \\ 
		& (0.028) & (0.051) & (0.025) & (0.047) \\ 
		& p = 0.00 & p = 0.00 & p = 0.00 & p = 0.00 \\ 
%		& & & & \\ 
		Observations & 422 & 422 & 422 & 422 \\ 
		R$^{2}$ & 0.033 & 0.081 & 0.020 & 0.086 \\ 
	\end{tabular} 
\end{table}  
	\end{block}

\end{frame}

%-----------------------------------------------------------------------------------
	\begin{frame}{Lab Results - Regression 1}

		\begin{block}{Notes}\small
			The table reports \textbf{linear regressions} on \texttt{inequality} in (columns (1)–(2) and on  \texttt{zero\_to\_worst\_off}
			(columns (3)–(4))

			Robust standard errors in
			parentheses.
		\end{block}


		\begin{block}{Example}\small
			\begin{itemize}
				\item Base (no choice) \texttt{inequality} = 0.204; \texttt{transfer} = 318.4 (NOK)
				\item Forced Choice \texttt{inequality} = (0.204+0.120); \texttt{transfer} = 270.4 (NOK)

				\texttt{inequality} $+60\%$ ($p = 0.007$)

				\item Nominal Choice \texttt{inequality} = (0.204+0.164); \texttt{transfer} = 252.8 (NOK)

				\texttt{inequality} $+80\%$ ($p = 0.001$)
			\end{itemize}

		\end{block}

	\end{frame}

\begin{frame}{Table 2: Heterogeneous effects in the lab experiment}
	
	\begin{block}\tiny
		\begin{table}[!htbp] \centering \tiny
	\caption{} 
	\label{tbl:l2} 
	\begin{tabular}{@{\extracolsep{4pt}}lcccccc} 
		\hline \\[-1.0ex] 
		\\[-1.0ex] & \multicolumn{6}{c}{inequality} \\ 
		\\[-1.0ex] & (1) & (2) & (3) & (4) & (5) & (6)\\ 
		\hline \\[-1.0ex] 
		choice & 0.144 & 0.258 & 0.250 & 0.157 & 0.105 & 0.361 \\ 
		& (0.037) & (0.058) & (0.053) & (0.055) & (0.054) & (0.098) \\ 
		& p = 0.001 & p = 0.00 & p = 0.00 & p = 0.005 & p = 0.054 & p = 0.001 \\ 
%		& & & & & & \\ 
		choiceTRUE:leftp &  & $-$0.192 &  &  &  & $-$0.146 \\ 
		&  & (0.074) &  &  &  & (0.075) \\ 
		&  & p = 0.010 &  &  &  & p = 0.052 \\ 
%		& & & & & & \\ 
		choiceTRUE:female &  &  & $-$0.235 &  &  & $-$0.216 \\ 
		&  &  & (0.073) &  &  & (0.085) \\ 
		&  &  & p = 0.002 &  &  & p = 0.012 \\ 
%		& & & & & & \\ 
		choiceTRUE:age\_h &  &  &  & $-$0.021 &  & $-$0.044 \\ 
		&  &  &  & (0.075) &  & (0.075) \\ 
		&  &  &  & p = 0.779 &  & p = 0.563 \\ 
%		& & & & & & \\ 
		choiceTRUE:crt\_h &  &  &  &  & 0.072 & $-$0.011 \\ 
		&  &  &  &  & (0.075) & (0.084) \\ 
		&  &  &  &  & p = 0.335 & p = 0.892 \\ 
		Constant & 0.312 & 0.240 & 0.232 & 0.303 & 0.338 & 0.162 \\ 
		& (0.051) & (0.056) & (0.059) & (0.059) & (0.058) & (0.078) \\ 
		& p = 0.00 & p = 0.00 & p = 0.00 & p = 0.00 & p = 0.00 & p = 0.038 \\ 
		Linear combination &   & 0.066 & 0.015 & 0.136 & 0.177 &  \\ 
		&  & (0.047) & (0.050) & (0.050) & (0.051) &  \\ 
		&  & p=0.160 & p=0.765 & p=0.007 & p=0.001 &  \\ 
		Other controls & Yes & Yes & Yes & Yes & Yes & Yes \\ 
		Observations & 422 & 422 & 422 & 422 & 422 & 422 \\ 
		R$^{2}$ & 0.080 & 0.093 & 0.100 & 0.080 & 0.082 & 0.109 \\ 
	\end{tabular} 
	\end{table}  
	\end{block}
	
\end{frame}
%------------------------------------------------------------------------------------
	\begin{frame}{Lab Experiment - Heterogeneous effects on Inequality}

	\begin{block}{Notes}{\small
		The table reports linear regressions of the variable “Inequality”, which includes interactions between being in one of the choice treatments and the background variables. “Choice”: indicator variable for the spectator being in the Nominal Choice or the Forced Choice treatment. The “Linear combination” row shows the treatment effect of choice on the group that has the value one on the corresponding background variable, while “Choice” shows the treatment effect for the other group. Robust standard errors in parentheses. %The estimates for the other controls are shown in Table A.2.
}
		\end{block}
\end{frame}
%------------------------------------------------------------------------------------
\begin{frame}{Online Experiment}

	\begin{block}{Setup}

		\begin{itemize}
			\item Participants
			\begin{itemize}
				\item 2,437 participants on Amazon Mechanical Turk (AMT)
				\item 8 USD or $\in (-0.25, 0, .25)$
				\item random allocation to: work + earnings/earnings only
			\end{itemize}
			\item Spectators
			\begin{itemize}
				\item 5,757 spectators from Norway (KANTAR)
				\item on average 48.5 years old, 52\% male, average 1.4/3 on CRT, and 31\% self-reported support for right-wing parties in Norway.
				\item spectators were randomly allocated to treatments and were paid a fixed compensation for taking part in the study, independent of their spectator decision.
				%\item Average payment was 475 NOK (approximately 80 USD) incl 100 basic fee.
			\end{itemize}
		\end{itemize}
	\end{block}
\end{frame}

%------------------------------------------------------------------------------------
\begin{frame}{Experimental Design - online Variables}

	\begin{block}{Treatment variables}\small
		\begin{itemize}
			\item \texttt{treatment} $\in$ (“Forced Choice”, “Nominal Choice”)
			\item \texttt{workp} TRUE if work required of participant
			\item \texttt{choice} TRUE if \texttt{treatment} $\in$ (“Forced Choice”, “Nominal Choice”)
		\end{itemize}
		\end{block}

	\begin{block}{Spectator Variables}\small
		\begin{itemize}
			\item \texttt{leftp}: self-reporting voting for a non-right-wing party in the last election
			\item \texttt{female}: TRUE if female
			\item \texttt{age\_h}: TRUE if age $\ge$ median (49)
			\item \texttt{crt\_h}: TRUE if score on cognitive reflection test $\ge$ median (2 out of 3).
			\item  \texttt{university} TRUE if university education
			\item \texttt{high\_income} TRUE if above median income ($>500,000$)
		\end{itemize}

		\end{block}
	\end{frame}

%-----------------------------------------------------------------------------------
\begin{frame}{Online Experiment Transfers}
	\includegraphics[height=.7\textheight]{../graphs/histograms_kantar_wd.pdf}
	{\tiny
	\emph{Note}:The figure shows histograms of the amount of money transferred from the lucky to
	the unlucky participant by the spectator in each treatment. The top two panels are for treatments
	with work requirements, the two bottom panels are for treatments without such a requirement.}
\end{frame}
%------------------------------------------------------------------------------------

\begin{frame}{Online Experiment Inequality}
	\includegraphics[height=.70\textheight]{../graphs/mean_ineq_nothing_kantar_wd.pdf}
	{\tiny
	\emph{Note}: The left panels show the average inequality implemented by the spectators in each treatment, the right panel shows the share of spectators assigning no income to one of the participants in the pair in each of the treatments. The top panels show for treatments with work requirements, the bottom panels show for treatments without such a requirement. The standard errors of the mean are indicated.
}
\end{frame}
%------------------------------------------------------------------------------------
%-----------------------------------------------------------------------------------
%-----------------------------------------------------------------------------------
\begin{frame}{Table 3: Online Results -  Regression Analysis: the role of choice}
	
	\begin{block}\tiny
		
	\begin{table}[!htbp] \centering \tiny
		\begin{tabular}{@{\extracolsep{0pt}}lcccccc} 
	\\[-1.0ex] & \multicolumn{3}{c}{inequality} & \multicolumn{3}{c}{zero\_to\_worst\_off} \\ 
	\\[-1.0ex] & (1) & (2) & (3) & (4) & (5) & (6)\\ 
	\hline \\[-1.0ex] 
	treatmentgroupForced Choice & 0.166 & 0.162 & 0.162 & 0.130 & 0.127 & 0.127 \\ 
	& (0.011) & (0.011) & (0.011) & (0.011) & (0.011) & (0.011) \\ 
	& & & & & & \\ 
	treatmentgroupNominal Choice & 0.109 & 0.108 & 0.108 & 0.066 & 0.066 & 0.066 \\ 
	& (0.013) & (0.013) & (0.013) & (0.012) & (0.012) & (0.012) \\ 
	& & & & & & \\ 
	workp & $-$0.068 & $-$0.069 & $-$0.069 & $-$0.058 & $-$0.059 & $-$0.059 \\ 
	& (0.011) & (0.010) & (0.010) & (0.010) & (0.010) & (0.010) \\ 
	leftp &  & $-$0.067 & $-$0.064 &  & $-$0.050 & $-$0.047 \\ 
	&  & (0.011) & (0.011) &  & (0.011) & (0.011) \\ 
	female &  & $-$0.094 & $-$0.084 &  & $-$0.054 & $-$0.044 \\ 
	&  & (0.010) & (0.010) &  & (0.010) & (0.010) \\ 
	age\_h &  & $-$0.077 & $-$0.078 &  & $-$0.056 & $-$0.057 \\ 
	&  & (0.010) & (0.010) &  & (0.009) & (0.009) \\ 
	crt\_h &  & 0.066 & 0.059 &  & 0.059 & 0.051 \\ 
	&  & (0.010) & (0.010) &  & (0.010) & (0.010) \\ 
	university &  &  & 0.016 &  &  & 0.021 \\ 
	&  &  & (0.010) &  &  & (0.010) \\ 
	high\_income &  &  & 0.050 &  &  & 0.049 \\ 
	&  &  & (0.011) &  &  & (0.011) \\ 
	& & & & & & \\ 
	Constant & 0.201 & 0.299 & 0.269 & 0.113 & 0.174 & 0.141 \\ 
	& (0.010) & (0.015) & (0.016) & (0.009) & (0.015) & (0.015) \\ 
	& & & & & & \\ 
	Observations & 5,757 & 5,757 & 5,757 & 5,757 & 5,757 & 5,757 \\ 
	R$^{2}$ & 0.033 & 0.080 & 0.084 & 0.023 & 0.049 & 0.054 \\ 
	\hline \\[-1.0ex] 
			\end{tabular} 
		\end{table}  
	\end{block}
	
\end{frame}

%-----------------------------------------------------------------------------------
\begin{frame}{Online Results - Regression 3}
	
	\begin{block}{Notes}\small
		The table reports linear regressions on \texttt{inequality} in (columns (1)–(3) and on \texttt{zero\_to\_worst\_off} (columns (4)–(6)).
		
		Robust standard errors in parentheses.
	\end{block}

	
	\begin{block}{Example}\small
		keeping workp constant
		\begin{itemize}
			\item Base (no choice) \texttt{inequality} = 0.201; \texttt{transfer} = 1.61 (USD)
			\item Forced Choice \texttt{inequality} = (0.201+0.166 = 0.367); \texttt{transfer} = 2.53 (USD)
			
			\texttt{inequality} $+83\%$ ($p = 0.011$)
			
			\item Nominal Choice \texttt{inequality} = (0.204+0.109); \texttt{transfer} = 2.76 (USD)
			
			\texttt{inequality} $+54\%$ ($p = 0.013$)
		\end{itemize}
		
	\end{block}
	
\end{frame}

%------------------------------------------------------------------------------------

\begin{frame}{Table 2: Heterogeneous effects in the lab experiment}
	
	\begin{block}\tiny
		\begin{table}[!htbp] \centering \tiny
  \caption{} 
\label{tbl:o2} 
\begin{tabular}{@{\extracolsep{0pt}}lcccccccc} 
	\\[-1.0ex] & \multicolumn{8}{c}{inequality} \\ 
	\\[-1.0ex] & (1) & (2) & (3) & (4) & (5) & (6) & (7) & (8)\\ 
	\hline \\[-1.0ex] 
	choice & 0.142 & 0.158 & 0.161 & 0.171 & 0.123 & 0.142 & 0.133 & 0.179 \\ 
	& (0.010) & (0.019) & (0.015) & (0.014) & (0.013) & (0.015) & (0.012) & (0.028) \\ 
%	& p = 0.00 & p = 0.00 & p = 0.00 & p = 0.00 & p = 0.00 & p = 0.00 & p = 0.00 & p = 0.00 \\ 
%	& & & & & & & & \\ 
	choice*Left &  & $-$0.023 &  &  &  &  &  & $-$0.017 \\ 
	&  & (0.022) &  &  &  &  &  & (0.023) \\ 
%	&  & p = 0.30 &  &  &  &  &  & p = 0.45 \\ 
%	& & & & & & & & \\ 
	choice*female &  &  & $-$0.038 &  &  &  &  & $-$0.027 \\ 
	&  &  & (0.020) &  &  &  &  & (0.021) \\ 
%	&  &  & p = 0.06 &  &  &  &  & p = 0.20 \\ 
%	& & & & & & & & \\ 
	choice*age &  &  &  & $-$0.064 &  &  &  & $-$0.061 \\ 
	&  &  &  & (0.020) &  &  &  & (0.020) \\ 
%	&  &  &  & p = 0.00 &  &  &  & p = 0.00 \\ 
%	& & & & & & & & \\ 
	choice*crt &  &  &  &  & 0.043 &  &  & 0.030 \\ 
	&  &  &  &  & (0.020) &  &  & (0.021) \\ 
%	&  &  &  &  & p = 0.03 &  &  & p = 0.16 \\ 
%	& & & & & & & & \\ 
	choice*uni &  &  &  &  &  & 0.002 &  & $-$0.006 \\ 
	&  &  &  &  &  & (0.020) &  & (0.021) \\ 
%	&  &  &  &  &  & p = 0.94 &  & p = 0.79 \\ 
%	& & & & & & & & \\ 
	choice*income &  &  &  &  &  &  & 0.028 & 0.018 \\ 
	&  &  &  &  &  &  & (0.022) & (0.024) \\ 
%	&  &  &  &  &  &  & p = 0.21 & p = 0.45 \\ 
%	& & & & & & & & \\ 
	workp & $-$0.060 & $-$0.060 & $-$0.060 & $-$0.060 & $-$0.060 & $-$0.060 & $-$0.060 & $-$0.060 \\ 
	& (0.010) & (0.010) & (0.010) & (0.010) & (0.010) & (0.010) & (0.010) & (0.010) \\ 
%	& p = 0.00 & p = 0.00 & p = 0.00 & p = 0.00 & p = 0.00 & p = 0.00 & p = 0.00 & p = 0.00 \\ 
%	& & & & & & & & \\ 
	Constant & 0.264 & 0.253 & 0.251 & 0.244 & 0.279 & 0.265 & 0.271 & 0.239 \\ 
	& (0.016) & (0.020) & (0.018) & (0.018) & (0.017) & (0.018) & (0.017) & (0.024) \\ 
%	& p = 0.00 & p = 0.00 & p = 0.00 & p = 0.00 & p = 0.00 & p = 0.00 & p = 0.00 & p = 0.00 \\ 
	Linear comb &   & 0.135 & 0.123 & 0.107 & 0.166 & 0.143 & 0.161 &  \\ 
	&  & (0.012) & (0.013) & (0.014) & (0.016) & (0.014) & (0.019) &  \\ 
	Observations & 5,757 & 5,757 & 5,757 & 5,757 & 5,757 & 5,757 & 5,757 & 5,757 \\ 
	R$^{2}$ & 0.081 & 0.081 & 0.082 & 0.082 & 0.082 & 0.081 & 0.081 & 0.083 \\ 
			\end{tabular} 
		\end{table}  
	\end{block}
	
\end{frame}
%-----------------------------------------------------------------------------------
%------------------------------------------------------------------------------------
\begin{comment}
%-----------------------------------------------------------------------------------


\begin{frame}{Table 3: Online Results -  Regression Analysis: the role of choice}{\tiny
		%TODO still too long
		% Table created by stargazer v.5.2.3 by Marek Hlavac, Social Policy Institute. E-mail: marek.hlavac at gmail.com
		% Date and time: Fri, Mar 31, 2023 - 18:10:09
		\input{../tables/main_online.tex}
	}  % end tiny
\end{frame}


\begin{frame}{Table 4: Online Results -  Regression 4: Heterogeneous effects on Inequality}{\tiny
		%TODO still too long
		% Table created by stargazer v.5.2.3 by Marek Hlavac, Social Policy Institute. E-mail: marek.hlavac at gmail.com
		% Date and time: Fri, Mar 31, 2023 - 18:10:09
		\input{../tables/online_2.tex}
	}  % end tiny
\end{frame}

\begin{frame}{Table 1: Lab Results Regression 1: the role of choice}{\tiny
		
		%	\begin{block}
			
% Table created by stargazer v.5.2.3 by Marek Hlavac, Social Policy Institute. E-mail: marek.hlavac at gmail.com
% Date and time: Sat, Apr 01, 2023 - 00:21:19
\begin{table}[!htbp] \centering 
  \caption{} 
  \label{tbl:l1} 
\begin{tabular}{@{\extracolsep{5pt}}lcccc} 
\\[-1.8ex]\hline 
\hline \\[-1.8ex] 
\\[-1.8ex] & \multicolumn{2}{c}{inequality} & \multicolumn{2}{c}{zero\_to\_worst\_off} \\ 
\\[-1.8ex] & (1) & (2) & (3) & (4)\\ 
\hline \\[-1.8ex] 
 treatmentForced Choice & 0.120 & 0.125 & 0.094 & 0.101 \\ 
  & (0.044) & (0.044) & (0.043) & (0.042) \\ 
  & p = 0.007 & p = 0.005 & p = 0.028 & p = 0.017 \\ 
  & & & & \\ 
 treatmentNominal Choice & 0.164 & 0.163 & 0.125 & 0.128 \\ 
  & (0.044) & (0.044) & (0.044) & (0.043) \\ 
  & p = 0.001 & p = 0.001 & p = 0.005 & p = 0.003 \\ 
  & & & & \\ 
 leftp &  & $-$0.115 &  & $-$0.075 \\ 
  &  & (0.037) &  & (0.037) \\ 
  &  & p = 0.003 &  & p = 0.044 \\ 
  & & & & \\ 
 female &  & $-$0.108 &  & $-$0.159 \\ 
  &  & (0.040) &  & (0.039) \\ 
  &  & p = 0.007 &  & p = 0.000 \\ 
  & & & & \\ 
 age\_h &  & 0.017 &  & 0.051 \\ 
  &  & (0.037) &  & (0.036) \\ 
  &  & p = 0.646 &  & p = 0.157 \\ 
  & & & & \\ 
 crt\_h &  & 0.001 &  & 0.009 \\ 
  &  & (0.040) &  & (0.039) \\ 
  &  & p = 0.984 &  & p = 0.827 \\ 
  & & & & \\ 
 Constant & 0.204 & 0.310 & 0.103 & 0.182 \\ 
  & (0.028) & (0.051) & (0.025) & (0.047) \\ 
  & p = 0.000 & p = 0.000 & p = 0.000 & p = 0.000 \\ 
  & & & & \\ 
Observations & 422 & 422 & 422 & 422 \\ 
R$^{2}$ & 0.033 & 0.081 & 0.020 & 0.086 \\ 
\hline \\[-1.8ex] 
\textit{Notes:} & \multicolumn{4}{l}{} \\ 
\end{tabular} 
\end{table}  

			
			%	\end{block}
		
	}  % end tiny
\end{frame}


\begin{frame}{Regression 3}
	
	\begin{block}{Notes}{\small
			The table reports linear regressions on \texttt{Inequality} columns 1–3, and  on  \texttt{zero\_to\_worst\_off}  columns 4–6
			
			Robust standard errors in parentheses.
		} % end tiny
	\end{block}
\end{frame}

\begin{frame}{Online Experiment - Heterogeneous effects on Inequality}

	\begin{block}{Notes}
		The table reports linear regressions of the variable “Inequality” on “Choice”: indicator variable for
		the spectator being in the Nominal Choice or Forced Choice treatment. “Left-wing”: indicator variable for
		the spectator self-reporting that he or she voted for a non-right-wing party in the last election. “Female”:
		indicator variable for the spectator being female. “Age”: indicator variable for the spectator's age being at
		or above the median in the sample (49 years). “Cognitive Reflection”: indicator variable for the spectator's
		score on the cognitive reflection test being at or above median (2 out of 3 points). “University education”:
		indicator variable for the spectator having university education. “High income”: indicator variable for the
		spectator having above median income (above 500 000 NOK). “Work requirement”: indicator variable for
		the participants being in a work requirement treatment. Robust standard errors in parentheses. %The full regression table is shown in Table A.2.
	\end{block}
\end{frame}
\end{comment}


%------------------------------------------------------------------------------------
\begin{comment}

%TODO maybe
\begin{frame}{Lab experiment - Political Affiliation}
	\includegraphics[height=.8\textheight]{../graphs/mean_ineq_nothing_robust_kantar.pdf}
	{\footnotesize
		\emph{Note}:The figure shows the distribution of political affiliations in the lab experiment
		and in the general population in the election in Norway prior to this study. SV:
		Sosialistisk Venstreparti; AP: Arbeiderpartiet; SP: Senterpartiet; Krf: Kristelig
		Folkeparti; V: Venstre; H: Høyre; Frp: Fremskrittspartiet. “Høyre” and “Fremskrittspartiet”
		are the two right-wing parties in Norway.}
\end{frame}
\end{comment}

%------------------------------------------------------------------------------------
\begin{frame}{Inequality implemented - forced choices}
	\includegraphics[height=.75\textheight]{../graphs/mean_ineq_nothing_robust_kantar.pdf}
	
	{\tiny 
		\emph{Note}:The left panel shows the average inequality implemented by the spectators in the base treatment and in each of the three forced choice treatments, the right panel shows the share of spectators assigning no income to one of the participants in the
		pair in each of these treatments. The standard errors of the mean are indicated.}
\end{frame}
%------------------------------------------------------------------------------------

\begin{frame}{Control over earnings?}
	\includegraphics[height=.70\textheight]{../graphs/control_earnings.pdf}
	{\tiny
		\emph{Note}: The figure shows the histogram of how spectators responded to the question
		of whether the participants had control over their earnings, by treatment. The question
		asked was: “Before you made your choice, participant A earned 8 USD, while participant B earned 0 USD. To what extent did the two participants have control over their own earnings before you made your choice?” The alternatives given were on a 1–7 scale, with 1 indicating “no control” and 7 indicating “full control.”
	}
\end{frame}
%------------------------------------------------------------------------------------
	\begin{frame}{Conclusions}

	\begin{block}

		\begin{quote}
			Our findings suggest that people consider the role of choice in determining agency
			and personal responsibility to go beyond the restrictions of the two minimal conditions.
			They find individual choices morally relevant in cases where these choices do
			not change the ex ante probabilities of the outcomes and in cases where there is no
			acceptable alternative to the chosen alternative. Our experimental results thus show
			that the presence of choice is a remarkably powerful source of inequality acceptance
			in society.
		\end{quote}

	\end{block}

\end{frame}


%------------------------------------------------------------------------------------
\begin{frame}{Lab Data - model transfer (scaled)}{\tiny
		%TODO still too long
		% Table created by stargazer v.5.2.3 by Marek Hlavac, Social Policy Institute. E-mail: marek.hlavac at gmail.com
		% Date and time: Fri, Mar 31, 2023 - 18:10:09
		
% Table created by stargazer v.5.2.3 by Marek Hlavac, Social Policy Institute. E-mail: marek.hlavac at gmail.com
% Date and time: Sun, Apr 02, 2023 - 18:08:20
\begin{table}[!htbp] \centering 
  \caption{} 
  \label{tbl:o:trans} 
\begin{tabular}{@{\extracolsep{5pt}}lcccc} 
\\[-1.0ex]\hline 
\hline \\[-1.0ex] 
\\[-1.0ex] & \multicolumn{2}{c}{inequality} & \multicolumn{2}{c}{transfer} \\ 
\\[-1.0ex] & (1) & (2) & (3) & (4)\\ 
\hline \\[-1.0ex] 
 treatmentForced Choice & 0.120 & 0.125 & $-$0.044 & $-$0.045 \\ 
  & (0.044) & (0.044) & (0.023) & (0.023) \\ 
  & p = 0.007 & p = 0.005 & p = 0.056 & p = 0.047 \\ 
%  & & & & \\ 
 treatmentNominal Choice & 0.164 & 0.163 & $-$0.038 & $-$0.038 \\ 
  & (0.044) & (0.044) & (0.024) & (0.024) \\ 
  & p = 0.001 & p = 0.001 & p = 0.114 & p = 0.115 \\ 
%  & & & & \\ 
 leftp &  & $-$0.115 &  & 0.026 \\ 
  &  & (0.037) &  & (0.021) \\ 
  &  & p = 0.003 &  & p = 0.213 \\ 
%  & & & & \\ 
 female &  & $-$0.108 &  & 0.035 \\ 
  &  & (0.040) &  & (0.021) \\ 
  &  & p = 0.007 &  & p = 0.103 \\ 
%  & & & & \\ 
 age\_h &  & 0.017 &  & $-$0.008 \\ 
  &  & (0.037) &  & (0.020) \\ 
  &  & p = 0.646 &  & p = 0.683 \\ 
%  & & & & \\ 
 crt\_h &  & 0.001 &  & 0.001 \\ 
  &  & (0.040) &  & (0.021) \\ 
  &  & p = 0.984 &  & p = 0.958 \\ 
%  & & & & \\ 
 Constant & 0.204 & 0.310 & 0.398 & 0.371 \\ 
  & (0.028) & (0.051) & (0.014) & (0.026) \\ 
  & p = 0.000 & p = 0.000 & p = 0.000 & p = 0.000 \\ 
%  & & & & \\ 
Observations & 422 & 422 & 422 & 422 \\ 
R$^{2}$ & 0.033 & 0.081 & 0.009 & 0.022 \\ 
\hline \\[-1.0ex] 
\end{tabular} 
\end{table}  

	}  % end tiny
	
\end{frame}

\begin{frame}{Online Data - model transfer (scaled)}{\tiny
		%TODO still too long
		% Table created by stargazer v.5.2.3 by Marek Hlavac, Social Policy Institute. E-mail: marek.hlavac at gmail.com
		% Date and time: Fri, Mar 31, 2023 - 18:10:09
		
\begin{table}[!htbp] \centering 
  \caption{} 
  \label{tbl:o:trans} 
\begin{tabular}{@{\extracolsep{5pt}}lccccc} 
\\[-1.8ex]\hline 
\hline \\[-1.8ex] 
\\[-1.8ex] & \multicolumn{2}{c}{inequality} & \multicolumn{3}{c}{transfer} \\ 
\\[-1.8ex] & (1) & (2) & (3) & (4) & (5)\\ 
\hline \\[-1.8ex] 
 treatmentgroupForced Choice & 0.166 & 0.162 & $-$0.077 & $-$0.075 & $-$0.075 \\ 
  & (0.011) & (0.011) & (0.011) & (0.006) & (0.006) \\ 
  & p = 0.000 & p = 0.000 & p = 0.000 & p = 0.000 & p = 0.000 \\ 
  & & & & & \\ 
 treatmentgroupNominal Choice & 0.109 & 0.108 & $-$0.044 & $-$0.044 & $-$0.045 \\ 
  & (0.013) & (0.013) & (0.013) & (0.007) & (0.007) \\ 
  & p = 0.000 & p = 0.000 & p = 0.001 & p = 0.000 & p = 0.000 \\ 
  & & & & & \\ 
 workp & $-$0.068 & $-$0.069 & 0.035 & 0.035 & 0.035 \\ 
  & (0.011) & (0.010) & (0.010) & (0.006) & (0.006) \\ 
  & p = 0.000 & p = 0.000 & p = 0.001 & p = 0.000 & p = 0.000 \\ 
  & & & & & \\ 
 leftp &  & $-$0.067 &  & 0.033 & 0.032 \\ 
  &  & (0.011) &  &  & (0.006) \\ 
  &  & p = 0.000 &  &  & p = 0.000 \\ 
  & & & & & \\ 
 female &  & $-$0.094 &  & 0.033 & 0.030 \\ 
  &  & (0.010) &  &  & (0.005) \\ 
  &  & p = 0.000 &  &  & p = 0.000 \\ 
  & & & & & \\ 
 age\_h &  & $-$0.077 &  & 0.033 & 0.033 \\ 
  &  & (0.010) &  &  & (0.005) \\ 
  &  & p = 0.000 &  &  & p = 0.000 \\ 
  & & & & & \\ 
 crt\_h &  & 0.066 &  & $-$0.052 & $-$0.047 \\ 
  &  & (0.010) &  &  & (0.005) \\ 
  &  & p = 0.000 &  &  & p = 0.000 \\ 
  & & & & & \\ 
 university &  &  &  &  & $-$0.019 \\ 
  &  &  &  &  &  \\ 
  &  &  &  &  &  \\ 
  & & & & & \\ 
 high\_income &  &  &  &  & $-$0.016 \\ 
  &  &  &  &  &  \\ 
  &  &  &  &  &  \\ 
  & & & & & \\ 
 Constant & 0.201 & 0.299 & 0.414 & 0.383 & 0.399 \\ 
  & (0.010) & (0.015) & (0.016) & (0.005) & (0.008) \\ 
  & p = 0.000 & p = 0.000 & p = 0.000 & p = 0.000 & p = 0.000 \\ 
  & & & & & \\ 
Observations & 5,757 & 5,757 & 5,757 & 5,757 & 5,757 \\ 
R$^{2}$ & 0.033 & 0.080 & 0.026 & 0.067 & 0.071 \\ 
\hline \\[-1.8ex] 
\textit{Notes:} & \multicolumn{5}{l}{} \\ 
\end{tabular} 
\end{table}  

	}  % end tiny
	% run anova?
\end{frame}
%------------------------------------------------------------------------------------


\begin{comment}

\begin{frame}{Table 2: Lab Results - Heterogeneous effects on Inequality}{\tiny
		%TODO still too long
		% Table created by stargazer v.5.2.3 by Marek Hlavac, Social Policy Institute. E-mail: marek.hlavac at gmail.com
		% Date and time: Fri, Mar 31, 2023 - 18:10:09
		
% Table created by stargazer v.5.2.3 by Marek Hlavac, Social Policy Institute. E-mail: marek.hlavac at gmail.com
% Date and time: Sat, Apr 01, 2023 - 00:21:42
\begin{table}[!htbp] \centering 
  \caption{} 
  \label{tbl:l2} 
\begin{tabular}{@{\extracolsep{5pt}}lcccccc} 
\\[-1.8ex]\hline 
\hline \\[-1.8ex] 
\\[-1.8ex] & \multicolumn{6}{c}{inequality} \\ 
\\[-1.8ex] & (1) & (2) & (3) & (4) & (5) & (6)\\ 
\hline \\[-1.8ex] 
 choice & 0.144 & 0.258 & 0.250 & 0.157 & 0.105 & 0.361 \\ 
  & (0.037) & (0.058) & (0.053) & (0.055) & (0.054) & (0.098) \\ 
  & p = 0.001 & p = 0.000 & p = 0.000 & p = 0.005 & p = 0.054 & p = 0.001 \\ 
  & & & & & & \\ 
 choiceTRUE:leftp &  & $-$0.192 &  &  &  & $-$0.146 \\ 
  &  & (0.074) &  &  &  & (0.075) \\ 
  &  & p = 0.010 &  &  &  & p = 0.052 \\ 
  & & & & & & \\ 
 choiceTRUE:female &  &  & $-$0.235 &  &  & $-$0.216 \\ 
  &  &  & (0.073) &  &  & (0.085) \\ 
  &  &  & p = 0.002 &  &  & p = 0.012 \\ 
  & & & & & & \\ 
 choiceTRUE:age\_h &  &  &  & $-$0.021 &  & $-$0.044 \\ 
  &  &  &  & (0.075) &  & (0.075) \\ 
  &  &  &  & p = 0.779 &  & p = 0.563 \\ 
  & & & & & & \\ 
 choiceTRUE:crt\_h &  &  &  &  & 0.072 & $-$0.011 \\ 
  &  &  &  &  & (0.075) & (0.084) \\ 
  &  &  &  &  & p = 0.335 & p = 0.892 \\ 
  & & & & & & \\ 
 leftp & $-$0.116 & 0.012 & $-$0.124 & $-$0.116 & $-$0.115 & $-$0.027 \\ 
  & (0.037) & (0.058) & (0.037) & (0.038) & (0.038) & (0.058) \\ 
  & p = 0.002 & p = 0.843 & p = 0.001 & p = 0.002 & p = 0.003 & p = 0.643 \\ 
  & & & & & & \\ 
 female & $-$0.109 & $-$0.116 & 0.052 & $-$0.108 & $-$0.113 & 0.036 \\ 
  & (0.040) & (0.040) & (0.061) & (0.040) & (0.040) & (0.070) \\ 
  & p = 0.007 & p = 0.004 & p = 0.400 & p = 0.008 & p = 0.006 & p = 0.610 \\ 
  & & & & & & \\ 
 age\_h & 0.018 & 0.018 & 0.029 & 0.032 & 0.020 & 0.057 \\ 
  & (0.037) & (0.036) & (0.037) & (0.059) & (0.037) & (0.060) \\ 
  & p = 0.622 & p = 0.621 & p = 0.427 & p = 0.587 & p = 0.586 & p = 0.338 \\ 
  & & & & & & \\ 
 crt\_h & $-$0.003 & $-$0.005 & 0.010 & $-$0.003 & $-$0.051 & 0.014 \\ 
  & (0.040) & (0.040) & (0.039) & (0.040) & (0.062) & (0.068) \\ 
  & p = 0.948 & p = 0.901 & p = 0.799 & p = 0.940 & p = 0.416 & p = 0.836 \\ 
  & & & & & & \\ 
 Constant & 0.312 & 0.240 & 0.232 & 0.303 & 0.338 & 0.162 \\ 
  & (0.051) & (0.056) & (0.059) & (0.059) & (0.058) & (0.078) \\ 
  & p = 0.000 & p = 0.000 & p = 0.000 & p = 0.000 & p = 0.000 & p = 0.038 \\ 
  & & & & & & \\ 
Linear combination &   & 0.066 & 0.015 & 0.136 & 0.177 &  \\ 
 &  & (0.047) & (0.050) & (0.050) & (0.051) &  \\ 
 &  & p=0.160 & p=0.765 & p=0.007 & p=0.001 &  \\ 
Observations & 422 & 422 & 422 & 422 & 422 & 422 \\ 
R$^{2}$ & 0.080 & 0.093 & 0.100 & 0.080 & 0.082 & 0.109 \\ 
\hline \\[-1.8ex] 
\textit{Notes:} & \multicolumn{6}{l}{} \\ 
\end{tabular} 
\end{table}  

	}  % end tiny
\end{frame}


	\begin{frame}[fragile]{Twist}

	\begin{block}

		The modelling for the \texttt{zero\_to\_worst\_off}  response variable was done using a linear regression, even though it is a binary variable (1 if either participant gets 0, 800 otherwise)

		I ran the two main regressions again using \texttt{glm(...family = binomial``logit'' )}.

		The results are in Table\ref{tbl:tweak1}
	0 degrees of freedom so no result  -figure how to compare results - use predict ?
		\begin{lstlisting}
			anova(t1noth1_l, t1noth1_lB, test = "Chisq")

			Analysis of Variance Table

			Model 1: zero_to_worst_off ~ treatment
			Model 2: zero_to_worst_off ~ treatment
			Res.Df    RSS Df Sum of Sq Pr(>Chi)
			1    419  59.81
			2    419 382.96  0   -323.15
		\end{lstlisting}

	\end{block}

	\end{frame}

%------------------------------------------------------------------------------------
	\begin{frame}[fragile]{Twist}

	\begin{block}

		\begin{lstlisting}
			anova(t1noth2_lB, t1noth2_l, test = "Chisq")
			Analysis of Deviance Table
			Model: binomial, link: logit
			Response: zero_to_worst_off
			Terms added sequentially (first to last)

			Df Deviance Resid. Df Resid. Dev  Pr(>Chi)
			NULL                        421     391.85
			treatment  2   8.8913       419     382.96  0.011730 *
			leftp      1   6.7786       418     376.18  0.009226 **
			female     1  21.6997       417     354.48 3.188e-06 ***
			age_h      1   2.2018       416     352.28  0.137846
			crt_h      1   0.0525       415     352.23  0.818694
			---
			Signif. codes:  0 '***' 0.001 '**' 0.01 '*' 0.05 '.' 0.1 ' ' 1
		\end{lstlisting}
	\end{block}

\end{frame}

\end{comment}
%-----------------------------------------------

\end{document}
